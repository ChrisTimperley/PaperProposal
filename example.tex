\documentclass{paper_proposal}
\begin{document}

% (working) paper title
% potential venues (with deadlines)

Hello

% what is the problem that you're solving?
\section{Problem}
Researchers spend lots of time coming up with ad-hoc research proposals
without explicitly planning out the paper that they're working towards.

% which works are related to the work you are proposing?
\section{Related Work}
\begin{itemize}
  \item A Framework for Theoretical Research Proposals. Plato et al. 2016.
\end{itemize}

% what contributions are made by this paper?
\section{Contributions}
\begin{itemize}
  \item A \LaTeX class for crafting research paper proposals
\end{itemize}

%
\section{Claims to Novelty}
\begin{itemize}
  \item Absolutely none; I'm sure someone else has already had the idea of
    writing a \LaTeX class for their paper proposals.
\end{itemize}

\section{Threats to Validity}

% what are the threats to validity?
\subsection{Identification}
\begin{itemize}
  \item This example may be too simple to demonstrate the awesomeness of using
    \LaTeX classes.
\end{itemize}

% how are you mitigating them?
\subsection{Mitigation}
\begin{itemize}
  \item Trying 
\end{itemize}

% how could the work be extended in the future?
\section{Future Work}
\begin{itemize}
  \item To make this template look prettier
  \item To improve readability by grouping sections
\end{itemize}

% high-level structure of the paper
\section{Structure}

\begin{enumerate}
  \item Introduction
  \item Background
  \item Conclusion \& Future Work
\end{enumerate}

\end{document}
